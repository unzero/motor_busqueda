\documentclass[a4papper,10pt]{article}
\usepackage[spanish]{babel}
%\usepackage[utf8]{inputenc}
\usepackage{anysize}
\usepackage{hyperref}
\title{Motor de Busqueda}
\marginsize{2cm}{2cm}{2cm}{2cm}


\begin{document}
\maketitle

Con el fin de dar orden en el desarrollo se establecen las siguientes reglas:

\begin{itemize}
	\item El control del proyecto se realizara por medio de Github
	\item blbl
\end{itemize}

EL cronograma del proyecto es el siguiente:

\begin{itemize}
	\item Semana 1: Interfaz grafica, base de datos y conexi\'on Java-MySql	
	\item Semana 2: Funciones de busqueda y ranking de p\'aginas.
	\item Semana 3: Agregado de p\'aginas al motor.
\end{itemize}

A medida que se llegue a la semana se daran las instrucciones del trabajo a realizar. Igualmente las revisiones seran los dias viernes y ese mismo dia en el repositorio (\url{https://github.com/unzero/motor_busqueda}) se subiran lo de la siguiente semana.

\section{Semana 1}

En esta semana se debe dise\~nar la interfaz gr\'afica. Con ayuda de netbeans se deben poner men\'us y controles para las funciones basicas dentro de un motor de busqueda, se puede usar google como fuente de inspiraci\'on.

Otro de los puntos a desarrollar es la base de datos, para ello se usar\'a MySql como SGBD preferido la base de datos deber\'a tener los campos y la estructura adjunta en el diagrama Entidad Relaci\'on. Todo el c\'odigo para la creaci\'on y llenado inicial de la base de datos deber\'a estar en un archivo llamado Database.sql.

Por \'ultimo se debe realizar la conexi\'on entre Java-MySql para ello se debe usar los Controladores que nos brinda MySql y se debe desarrollar una clase Conexion.java la cual recibe el nombre de la base de datos y debe conectar; si no es posible realizar la conexi\'on la clase debe lanzar una excepci\'on hacia arriba, es decir debe arrojar una excepci\'on al punto de llamado. Para ello 


nombre\_metod(parametros..) throws Exception{};

Las fuentes incluyen google, Deitel, etc.


\end{document}
